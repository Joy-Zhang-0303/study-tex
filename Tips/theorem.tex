% いちいち説明するのめんどいので体で覚えるのじゃ
\documentclass[a4j]{jsarticle}
% theorem系の使用にはamsthm
\usepackage{amsmath, amssymb, amsthm}

% theorem系環境の設定を弄れる。
\newtheoremstyle{plain}% style name
{16pt}% above space
{16pt}% below space
{}% body font
{}% indent
{\bfseries}% head font
{}% head punctuation
{\newline}% head space
{}% head spec

% thoremstyleの選択
% 随時変更できます
\theoremstyle{plain}

% 引数なしproof環境でのデフォルト表示
\renewcommand{\proofname}{証明}

% theorem系環境の設定
% \newtheorem{envname}{表示名}[親カウンタ]
% \newtheorem{envname}[同期カウンタ]{表示名}
% 親カウンタを指定した場合はカウンタに親カウンタが付属します
% カウンタリセットも親カウンタと同タイミング
% 同期カウンタを指定した場合は同期カウンタと同じカウンタを使います。
\newtheorem{definition}{定義}[section]
\newtheorem{axiom}{公理}[section]
\newtheorem{theorem}{定理}[section]
\newtheorem{proposition}[theorem]{命題}
\newtheorem{lemma}[theorem]{補題}
\newtheorem{corollary}[theorem]{系}
% ここでは定義、公理、定理はセクションカウンタが付属し、
% セクションごとにリセット。
% 定理、命題、補題、系は同期しているので
% 定理1.1のあとに命題を設定すると命題1.2となる


\begin{document}
\section{定義とか公理とか}
\begin{axiom}[コルモゴロフの確率の公理]
  確率空間\((\Omega, F, P)\)について、
\begin{enumerate}
  \item \(P(E) \in \mathbb{R}, P(E) \geq 0 \hspace{16pt} \forall E \in F\)
  \item \(P(\Omega) = 1\)
  \item 可算個の排反事象\(E_1, E_2, \ldots\)について、
  \begin{align*}
    P\left(\bigcup_{i=1}^{\infty}E_i\right) &= \sum_{i=1}^{\infty}P(E_i)
  \end{align*}
\end{enumerate}
  となる。
\end{axiom}

\begin{definition}[周辺確率: Marginal probability]
  \label{def:marginal_p}
  事象\(A\)が発生する確率を事象\(A\)についての周辺確率といい、
  \(P(A)\)で表す。
\end{definition}
\begin{definition}[結合確率: Joint probability]
  \label{def:joint_p}
  事象\(A, B\)がどちらも発生する確率を事象\(A, B\)についての結合確率といい、
  \(P(A \cap B)\)または\(P(A, B)\)で表す。
\end{definition}
\begin{definition}[条件付き確率: Conditional probability]
  \label{def:conditional_p}
  事象\(B (P(B) > 0) \)が起こるという条件のもとで
  事象\(A\)が起こる確率を条件付き確率といい、
  \(P(A \mid B)\)または\(P_B(A)\)で表す。
  条件付き確率\(P(A \mid B)\)は以下の式で定義される。
\begin{align*}
  P(A \mid B) &= \dfrac{P(A \cap B)}{P(B)}
\end{align*}
\end{definition}

\begin{proposition}
  \(A, B \in F\)が\(A \subset B\)のとき、\(P(A) \geq P(B)\)となる。
\end{proposition}
\begin{proof}
\begin{align*}
  A &= B \cup (A \cap B^c)
  && \because A \subset B \\
  P(A) &= P(B \cup (A \cap B^c)) \\
  P(A) &= (P(B) + P(A \cap B^c)) - P(B \cap (A \cap B^c)) \\
  P(A) &= P(B) + P(A \cap B^c)
  && \because B \cap B^c = \phi \\
\end{align*}
確率の公理1より確率は非負であるため、
\begin{align*}
  P(A) &\geq P(B)
\end{align*}
\end{proof}

\begin{theorem}[ベイズの定理]
  \label{thm:bayes}
  事象Aについて、
  \begin{itemize}
    \item 事前確率を\(P(A)\)
    \item 事象\(B\)が起きた後の事後確率を\(P(A \mid B)\)
  \end{itemize}
  とするとき、事後確率\(P(A \mid B)\)は以下のように表せる。
\begin{align*}
  P(A \mid B) &= \dfrac{P(B \mid A)P(A)}{P(B)}
\end{align*}
\end{theorem}

\begin{proof}
\begin{align*}
  P(A \mid B) &= \dfrac{P(A \cap B)}{P(B)}
  && \because 定義\ref{def:conditional_p}\\
  P(A \cap B) &= P(A \mid B)P(B) \\
  P(B \cap A) &= P(B \mid A)P(A) \\
  P(A \mid B) &= \dfrac{P(B \mid A)P(A)}{P(B)} \\
\end{align*}
\end{proof}
\end{document}
